\chapter{Available ATC positions}
\section{Approach [APP]}
The Approach Controller has the following responsibilities:
\begin{itemize}
    \item controlling aircraft within TMA ond delegated airspaces,
    \item guiding arriving aircraft to their final approach,
    \item segregation of departing traffic,
    \item separating traffic according to the airspace class,
    \item coordination with other ATC: TWR, DIR, ACC,
    \item provision of Flight Information Service to traffic below the controlled airspace, within the horizontal boundaries of the airspace,
    \item providing "top-down" coverage of DIR position if it is offline or not provided at the aerodrome.
\end{itemize}

Arriving aircraft are handed over using silent coordination to the approach controller according to established standard arrival conditions (e.g., at a certain point or altitude) or according to coordination for particular traffic.

\section{Final Director [F\_APP]}
If a Director position is established in a given airspace, it is responsible for:
\begin{itemize}
    \item guiding aircraft to final descent and onto a stabilized final approach track based on published procedures,
    \item close coordination with "Approach" controller to build appropriate arrival sequence,
    \item provide "top-down" coverage of TWR position if it is offline.
\end{itemize}

The default area of responsibility boundary between the Approach controller and the Director in FIR Warszawa is FL90 unless local procedures say otherwise.

\section{Procedural Tower [TWR]}
When procedural control is exercised at the airport in TMA, by default, it is exercised by the Tower controller (TWR).

\section{Precision [P\_APP]}
At Poznań-Krzesiny Airport (EPKS), the ``Krzesiny Precision / Precision [EPKS\_P\_APP]'' position has been established. It is responsible for controlling aircraft on the precision radar approach (PAR).

The purpose of a precision approach using PAR radar is to enable the crew to make a safe landing by obtaining visual contact with at least one element of the runway environment at or before reaching a decision altitude, in a position enabling crew to continue the approach visually.

Precision approach control can only be conducted against a single aircraft. Due to
the need to maintain constant one-way communication between the PAR controller and the aircraft, PAR controller must not cover TWR position ``top-down''.

After establishing radio communication, PAR controller informs the crew of:
\begin{itemize}
    \item radar identification,
    \item type of conducted approach and runway direction,
    \item present QNH (if changed),
    \item glidepath angle,
    \item Obstacle Clearance Altitude (OCA) of the approach
\end{itemize}
and the necessity to check crew's minima.

\paragraph{Example:} \textit{SPABC, Krzesiny Precision, radar contact. This will be Precision Radar Approach, runway 29, [QNH 1014], expect 3 degrees glidepath, OCA 554~ft, check your minima.}

During the approach, the controller gives the crew  the following information in regular intervals (not less than once every 5 seconds):
\begin{itemize}
    \item distance to touchdown,
    \item aircraft position in regard to the extended runway centerline,
    \item aircraft position in regard to the glidepath,
    \item when necessary, information about the trend of changes in reported parameters.
\end{itemize}

\paragraph{Example:} \textit{5 miles from touchdown. Closing slowly from the left, heading is good. Slightly below glidepath.}

In order to relieve the load on the crew and to ensure the continued ability to transmit, the PAR controller issues a \textit{``do not acknowledge further transmissions''} instruction. From this point on, all PAR controller's instructions do not require a readback, except for landing clearance, instructions to go around, requesting a radio check.

The distance to touchdown should be given in 1 NM increments until the aircraft reaches a distance of 4 NM to the touchdown point. From 4 NM onward, distance information should be given more frequently, maintaining priority to information on elevation, direction and guidance.

Aircraft performing a PAR approach should be reminded during the final approach
to check landing gear (\textit{``check gear down and locked''}).

When justified, PAR controller may instruct the aircraft to:
\begin{itemize}
    \item change heading,
    \item adjust rate of descent,
    \item maintain level flight,
    \item go around.
\end{itemize}
PAR controller issues heading change instructions with 1\degree~resolution. Instructions to adjust rate of descent are given by describing the position of aircraft in regard to the glidepath (e.g. \emph{``well above glidepath''}) and issuing the instruction \emph{``adjust rate of descent''}.