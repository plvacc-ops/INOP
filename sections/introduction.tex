\chapter{Introduction}%
\label{ch:introduction}
\section{Document's Purpose}
The following document was created to establish guidelines and standardize operational procedures for Polish VACC virtual air traffic controllers as part of virtual air traffic control on the VATSIM network.

The document was created solely for the needs of the VATSIM network and cannot be used outside it, in particular it should not be used operationally within real air traffic control services.

\section{Document's Contents}
You should learn from and understand this document as follows:
\begin{itemize}
\item \textbf{general information} relating to specific types of control (aerodrome traffic control, approach control, radar and procedural control procedures),
\item \textbf{detailed information} relating to individual TMAs. The included information is structured as follows:
\begin{itemize}
\item \textbf{information about airports} within the TMA,
\item \textbf{information about the TMA airspace}
\end{itemize}
\item \textbf{attachments}, which mainly contain collected information in the form of Quick Reference Cards (QRCs), which are used to quickly view the most important information while exercising control.
\end{itemize}

\section{Definitions}
Expressions used in this document have the following meanings:
\begin{description}
    \item[Air Traffic Controller] --- \textit{(Controller, ATC)} --- a person responsible for the air traffic control service on the VATSIM network, issued a controller rating, allowed to control a selected position and logged in in accordance with the VATSIM Global Rating Policy.
    \item[Crew/Pilot] --- a person responsible for controlling the aircraft on the VATSIM network, connected in accordance with the VATSIM network rules.
\end{description}

\section{Legal Basis}
This document was created on the basis of the following legal bases, used and formatted for the needs of the VATSIM network:
\begin{itemize}
    \item ICAO Doc 4444 --- Procedures for Air Navigation Services, Air Traffic Management;
    \item AIP Polska;
    \item Polish VACC Policy;
    \item VATSIM Code of Conduct;
    \item VATSIM Code of Regulations;
    \item VATSIM Global Ratings Policy;
    \item VATSIM Global Controller/ATIS Information Policy;
    \item VATEUD Policies and Regulations.
\end{itemize}

\section{Content Liability}
The document is edited and updated by the Polish VACC Board. The main responsible for the document is the Member of the PL-VACC Board responsible for operational changes in vFIR Warszawa or --- in the absence thereof --- the Director of Polish VACC\@.

\section{ATC Responsibilities}
Pursuant to the provisions of Art. 4 Polish VACC Policy, especially point 2a of this article, person providing control at vFIR Warszawa is obliged to follow the procedures set by the relevant members of the Polish VACC Board, therefore knowledge of this document and its application in practice within the scope of their positions is mandatory.