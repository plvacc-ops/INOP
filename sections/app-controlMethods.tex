\chapter{Methods of control}

\section{Radar service}

The radar service is based on the use of surveillance system imaging with identified aircraft in order to ensure:
\begin{itemize}
    \item aircraft separation,
    \item air traffic monitoring, in order to inform about route deviations,
    \item radar vectoring to avoid traffic or shorten the route,
    \item assistance for aircraft in distress,
    \item coordination of different types of air traffic,
\end{itemize}
additionally, in case of radar service in approach control service:
\begin{itemize}
    \item radar vectoring to a position, from which final instrument approach can be conducted,
    \item radar vectoring to a position, from which visual approach can be conducted,
    \item monitoring instrument approach procedures and visual approaches.
\end{itemize}

In vFIR Warszawa, the following radar separation minima are applied:
\begin{description}
    \item[Horizontal separation] 5.0~NM
    \item[Vertical separation]\parbox[t]{0.8\textwidth}{
              below FL280: 1000~ft\\
              above FL280: 2000~ft, except RVSM airspace, as defined in section \ref{sssec:airspace:rvsm}}
\end{description}

\subsubsection{APP with surveillance capabilities}
In vFIR Warszawa, radar service is available at the following positions:
\begin{itemize}
    \item APP Gdańsk,
    \item APP Kraków,
    \item APP Poznań.
    \item APP Warszawa.
\end{itemize}

\subsubsection{Reduced lateral separation}
In following TMAs, approach may reduce lateral separation to 3~NM in 30~km (16~NM) radius from the radar antenna:
\begin{itemize}
    \item TMA Gdańsk,
    \item TMA Warszawa.
\end{itemize}

\subsubsection{Wake turbulence separation}
\begin{table}[htbp]
    \centering
    \begin{tabular}{|M{3cm}|M{3cm}|M{3cm}|}
        \hline\rowcolor{vred}
        \color{white}\textbf{Preceding} & \color{white}\textbf{Succeeding} & \color{white}\textbf{Separation} \\\hline
        SUPER (J)                       & \multirow{2}{*}{HEAVY (H)}       & 5 NM                             \\\cline{1-1}\cline{3-3}
        HEAVY (H)                       &                                  & 4 NM                             \\\hline
        SUPER (J)                       & \multirow{2}{*}{MEDIUM (M)}      & 7 NM                             \\\cline{1-1}\cline{3-3}
        HEAVY (H)                       &                                  & 5 NM                             \\\hline
        SUPER (J)                       & \multirow{3}{*}{LIGHT (L)}       & 8 NM                             \\\cline{1-1}\cline{3-3}
        HEAVY (H)                       &                                  & 6 NM                             \\\cline{1-1}\cline{3-3}
        MEDIUM (M)                      &                                  & 5 NM                             \\\hline
    \end{tabular}
    \caption{Wake turbulence separation}
    \label{tab:wtc_radar}
\end{table}

\subsubsection{Beginning and termination of radar service}
In order to begin radar service, aircraft must be identified.

Aircraft identification is described in ICAO Doc 4444: PANS-ATM, chapter 8, sections 8.6.2 and 8.6.3~\cite{4444}.

There are two main types of radar identification, depending on available equipment:

\paragraph{Primary Surveillance Radar identification:} \cite[sect. 8.6.2.4]{4444}
\begin{itemize}
    \item by correlating a particular radar position indication with an aircraft reporting its position over, or as bearing and distance from, a point displayed on the radar map, and by ascertaining that the track of the particular radar position is consistent with the aircraft path or reported heading;
    \item by correlating an observed radar position indication with an aircraft which is known to have just departed, provided that the identification is established within 2 km (1 NM) from the end of the runway used. Particular care should be taken to avoid confusion with aircraft holding over or overflying the aerodrome, or with aircraft departing from or making a missed approach over adjacent runways;
    \item  by ascertaining the aircraft heading, if circumstances require, and following a period of track observation:
          \begin{itemize}
              \item instructing the pilot to execute one or more changes of heading of 30 degrees or more and correlating the movements of one particular radar position indication with the aircraft's acknowledged execution of the instructions given; or
              \item correlating the movements of a particular radar position indication with manoeuvres currently executed by an aircraft having so reported.
          \end{itemize}
\end{itemize}

\paragraph{Secondary Surveillance Radar identification:} \cite[sect. 8.6.2.3]{4444}
\begin{itemize}
    \item  recognition of the aircraft identification in a radar label;
    \item  recognition of an assigned discrete code, the setting of which has been verified, in a radar label;
    \item direct recognition of the aircraft identification of a Mode S-equipped aircraft in a radar label;
    \item observation of compliance with an instruction to set a specific code;
    \item observation of compliance with an instruction to squawk IDENT;
\end{itemize}

In any case of identification, there must be reasonable assurance that there is no possibility of mistaking the traffic for another aircraft performing under similar flight conditions (e.g., same area, duplicate transponder code etc.)

Crew should be informed about beginning the radar service by using the phrase \textit{``identified''} or \textit{``radar contact''}.

Radar service termination may be conducted when:
\begin{itemize}
    \item aircraft exit airspace in which radar service is provided or is trasferred to a unit that does not provide radar service;
    \item aircraft descends below Minimum Vectoring Altitude (MVA);
    \item identification has been lost or there is reasonable certainty that the identification may be lost soon (e.g.~disappearing from scope and reappearing with a different squawk code);
    \item radar contact is lost;
    \item aircraft has landed.
\end{itemize}

The termination of radar service should be immediately communicated to the crew using the phrase \textit{``radar service terminated''}, except when the aircraft has landed.