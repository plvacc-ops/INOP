
\chapter{Procedures}
\label{chap:acc:procedures}

\subsubsection{Minimum radar separation}

\begin{description}
\item[Horizontal separation] 5.0~NM
\item[Vertical separation]\parbox[t]{0.8\textwidth}{ below FL280: 1000~ft\\
above FL280: 2000~ft, except RVSM airspace, as defined in \cref{sssec:airspace:rvsm}.}
\end{description}

\subsubsection{Silent coordination and transfer}

Same rules apply as for approach control. See \cref{sec:app:silentcoor}.

\subsubsection{Initial contact}

On initial contact ACC Warszawa:
\begin{itemize}
\item verifies the assigned code is squaked by the aircraft or assigns correct
  squawk code,
\item verifies radar identification or identifies aircraft.
\end{itemize}

\subsubsection{Arrival information}

ACC Warszawa issues a STAR clearance when one of the following conditions is
met:

\begin{itemize}
\item aircraft's arrival routes via TMA Warszawa (destination aerodrome: EPWA,
  EPMO, EPRA, EPLL),
\item next controller requested direct flight to a point on a STAR,
\item next controller requested to relay STAR clearance.
\end{itemize}

\subsubsection{Transition level setting procedure}

If at one of the controlled airports in FIR Warszawa, the current QNH drops
below 995 hPa, the ACC controller sets the transition level in FIR Warszawa to
FL~90.

If at all controlled airports in FIR Warszawa the current QNH is equal to or
greater than 995 hPa, the ACC controller sets the transition level to FL~80.

%%% Local Variables:
%%% mode: latex
%%% TeX-master: "../main"
%%% End:
